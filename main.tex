\documentclass[preprint,12pt]{elsarticle} 
\usepackage[utf8]{inputenc}
\makeatletter
\def\ps@pprintTitle{%
   \let\@oddhead\@empty
   \let\@evenhead\@empty
   \let\@oddfoot\@empty
   \let\@evenfoot\@oddfoot
}
\makeatother
\title{CoDesign of Learning Analytics Dashboards}
\author{...}
\date{January 2021}
 
\usepackage{natbib}
\usepackage{graphicx}
\usepackage[hmargin=3cm]{geometry}  
\linespread{1.3}
\begin{document}

\maketitle

\section{Introduction} 
The design and evaluation of learning dashboards is a major area of inquiry in learning analytics research.
Learning analytics dashboards are designed to use learners' traces to present the computed indicators and other visual elements in a clear and intuitive way \citep{Brouns2015}.
They have emerged as applications for visualizing and interacting with data collected in a learning environment in various forms \citep{Ramos2015}. \citet{Steiner2014} referred to them as ``visualizations of learning traces''. For \citet{Yoo2015}, a learning dashboard is ``a display which visualizes the results of educational data mining in a useful way''. 
\citet{Schwendimann2017} identified a lack of an agreed and shared definition and thus proposed the following:
\begin{quotation}
	``\textit{%
		A learning dashboard is a single display that aggregates different indicators
		about learner(s), learning process(es) and/or learning context(s) into one or
		multiple visualizations.
	}'' \citep{Schwendimann2017}
\end{quotation}
Learning dashboards provide interactive, historical, customized and analytical displays that are based on the results of  analyzing learning data \citep{Park2015, Kim2015}. By implementing visual and interactive analytics, they amplify human natural abilities to detect patterns, establish connections and make inferences. The produced visual outputs can significantly highlight aspects of interest from the mined and discovered knowledge~\citep{Duval2011}. 

Learning dashboards are suitable for online,  face-to-face, and blended learning \citep{Verbert2013}. They can target different stakeholders: administrators, instructors, learners or all of them. 
Within a single display, indicators and visualizations about learners, learning processes and contexts are rendered using different shapes, from plain text to visual elements (e.g., tables, spreadsheet charts, scatterplot, 3D representations) to complex artifacts such as alerts and notifications that prompt interventions~\citep{Few2006, Podgorelec2011, Schwendimann2017}. 
Currently, they are increasingly deployed as a meaningful component in learning analysis systems. For instance, they are currently used in studying progression through courses \citep{Nicholson2012}, learners level of attainment~\citep{Gutierrez2012}, and learners' engagement from the cognitive and behavioral perspectives \citep{Carrillo2017}.
Despite being fairly recent educational tools, the research found many benefits of using learning dashboards to improve learning performance \citep{Arnold2012} and to increase learners' motivation    \citep{Verbert2013, Wise2016}.

\dots


According to \citet{bennett2019four}, we can distinguish  three types of learner dashboards: predictive, modeler and descriptive. 
\begin{itemize}
    \item Predictive types use machine learning based on computer algorithms that rely on a series of trace data to predict the likelihood that the students outcomes. Many works in this category (i.e, \citep{arnold2014exercise}) use these algorithms to produce at risk rating for individual students.
    \item For example, aspects of a student's online behaviours such as communication, initiative, presence that have been derived from the number of messages, the number of comments in response to others' messages, time spent online, etc., are not included in this section. 
    \item The third type of dashboard is descriptive and displays past learning behaviours
\end{itemize}

\section{Designing leanring analytics dashboards}
There is a growing body of literature on learner dashboard design, and several literature reviews covering the field. A comprehensive literature review by \citet{bodily2017trends} involved identifying and analyzing 93 papers with respect to functionality, data sources, design analysis, student perceptions, and measured effects. They found that the clear majority of the papers focused on technical features such as data sources, functional aspects (e.g., visualization data, what the dashboard was intended to do) while only two papers reported on how students' behaviours were changed as a result of using the Learner Dashboards and only one-third (34 of 93) focused on students' perceptions of the design.
\subsection{General design principles}
Due to the recent emergence of learning analytics dashboards, there is still a scarcity of studies on their design principles \citep{Echeverria2018}. 
\citet{Yoo2015} argued that, since dashboards are an instrument of communication, effective design is tied to several theoretical foundations, such as human cognition and perception, situational awareness and visualization technologies. In other words, their conceptualization must be based on an understanding of how humans see and think. 

Based on a number of theoretical principles in addition to his practical experience, \citet{Few2013} outlined some good and bad examples of dashboard design. He claimed that the essential characteristics of a dashboard are: 
1) to be visual displays; 2) to display the information needed to achieve specific objectives, 3) to fit on a single computer screen, and 4) to be used to monitor information at a glance. 
In terms of human perception, due to the limited working memory of humans, only three or four pieces of visual information can be stored at a time.  Therefore, for more effective memory perception and retention, it is essential to incorporate graphic patterns such as graphs rather than individual numbers. In addition, there must be a proper and reasoned use of pre-attentive attributes such as colour, shape, spatial position and movement.
From Few's principles, \citet{Yoo2015} drew three main implications: 
\begin{enumerate}
	\item the most important information should stand out from the rest in a dashboard, which usually has limited space to fit into a single screen;
	\item the information in a dashboard should support one's situated awareness and help rapid perception using diverse visualization technologies; and
	\item the information should be deployed in a way that makes sense, and elements of information should support viewers' immediate goal and end goal for decision making.
\end{enumerate}

\textit{Situational awareness} deals with disclosing the type of information that is important for a particular purpose or task \citep{Endsley2016}, and thus constitutes another design principle related to dashboards. Three levels can be distinguished:
\begin{enumerate}
	\item perception of the elements in the environment;
	\item comprehension of the current situation; and
	\item projection of future status.
\end{enumerate}
\textit{Situational awareness} is commonly understood in terms of people being consistently aware of what is going on, in order to predict what will be happening as well as to prepare what must be done. 

\subsection{Principles of Students' dashboards}
Draws on an  interpretive lens of student agency and empowerment \cite{slade2013learning}, \citet{bennett2019four} identified four  principles to inform the design of learner dashboards. The key point of these principles is that they shift attention from the technical features of dashboard design to the forefront, rather than to how they are interpreted by students. In doing so, the principles emphasize the values of student engagement and empowerment that \citet{prinsloo2017big}  propose as a key principle for the ethical adoption of learning analysis. The four principles are: 
\begin{enumerate}
    \item \textit{designs that are  customisable by students}: this suggests practical ways that the function and form of dashboards might be designed so that students can tailor the displays to suit their particular needs which would reflect their aims and aspirations, their personal dispositions 
    \item \textit{foreground students sense making}: suggests making
    available the data sources so that students are able to interrogate these, and not aggregating sources so that their provenance is lost. It suggests enabling students to contextualise the displays to their particular programme and circumstances
    \item \textit{enable students to identify  actionable insights}: this suggests that designs should help students to interpret the data rather than digesting it for the student. This might mean providing guidance to the student about what is on view and prompts to help them to identify what actions they could take next.
    \item \textit{dashboard use is embedded into educational processes}: this places attention on the ways
    that the tools are integrated into the educational systems and processes. We know that uptake of learner dashboards has been low \cite{bodily2017trends} so a key focus in terms of the design is how it is embedded into student behaviours.
\end{enumerate}

\subsection{Limitation of the existing dashboards design approaches}
%TODO: COPY/PASTE from etd9277_SShiraziBeheshtiha.pdf

%Echeverria2018
\subsubsection{Lack of theoretically informed design}
%The majority of educational dashboards are based on theories and practices related to information and data visualization. 
One obstacle to the adoption of dashboards is the often existing gap between visual analyses and the objectives of the study  \citep{Roberts2017}. 
Sometimes, to represent the analyzed data from different angles, designers use complex representations and visualizations that are rather difficult for end users to interpret, especially "at a glance" \citep{Duval2011}. According to the survey reported in \citep{Reimers2015}, the existing dashboards generally have poor interface design and lack of usability testing. The selection of data to be visualized is generally not what the stakeholders in the analysis want or really need because they have not regularly been involved in the design process \citep{Holstein2017}.
\citet{Bodily2017} also noted the absence of design choice justifications in the conception of several learning dashboards. 

A primary concern of dashboard designers must be the identification of what type of visual representations to implement, and what kind of interaction to offer. 
\citet{Gavsevic2015} argue that, without careful considerations, the design of dashboards can result in the implementation of fragile and undesirable instructional practices by promoting ineffective feedback types and methods. 
In order to encourage adoption of learning dashboards, the design needs to be further informed by theories related to learning sciences and educational psychology. \citet{Holstein2017} argued that the success of the dashboards depends on the degree to which its stakeholders have been involved in co-designing them. 


\subsubsection{Selection of the input data and the computed indicators}
A rich variety of measured data and indicators are used and computed in existing dashboards. 
Dashboard solutions are heavily based on trace analysis, and little attention has been paid to use other data sources such as direct feedback or the quality of the produced artifacts.
Moreover, as noted by \citet{Schwendimann2017}, there is little work on comparing which indicators and which visualizations are most suitable for the different user data literacy levels.
In most cases, the chosen visualizations are rather similar to those in other areas of dashboard applications (e.g., web analytics),
which highlights the lack of specific visualizations and visual metaphors that address the activities of learning and teaching (another potential area for future research) \citep{Schwendimann2017}.

\subsubsection{Learners' aptitudes are not considered}
The review of \citet{Schwendimann2017} revealed that current focus hugely falls on the information that can be extracted from the log data rather than how individual differences might affect the interpretation of the information presented to the learner in terms of reflection and taking actions \citep{Gavsevic2015}. 

%here :
Studies undertaken by \citet*{bodily2017trends}  and \citet*{viberg2018current}  also point to  the need to understand users' views of dashboards qualitatively and in use rather than  as technical artefacts. This conception foregrounds the purpose of the tool in that it considers how  students make use of information presented via the dashboard and how it might change their  behaviour as a result. Dashboard conception needs also to focusse attention on the perspective of  the user, the learner, and on the purpose of dashboards as tools that provide feedback to students  to encourage them to make more informed decisions about their study behaviours \citet*{howell2018learning,bennett2019four}.


\section{CoDesign of learning analytics dashboard}
\subsection{User-centered design}
The goal of designing data visualizations is to empower users to better understand the data presented by the visualization, so the user must be kept in mind when designing data visualizations. Two concepts are relevant in this context: human-centered design and user-centered design :
\begin{itemize}
    \item \textit{Human-centred design} focuses on creating systems that are easier to use by taking human factors into account.
    \item \textit{User-centred design} involves users in the design process where design development is influenced by end-users. Model-based, story-based and scenario-based design are examples of user-centered design methodologies. 
\end{itemize}

For \citet{vilpola2008method}, user-centred design is an an imperative part of creating products that ensures usefulness to users, and that allows creation of systems that fullfill users requirements. The authors highlight four principles of user-centred design: (1) user is actively involved; (2) user's tasks and technological function must be distinguished; (3) the design should be iterative; and (4) the design should be multidisciplinary. 
\citet{vredenberg2001user}, while recongnizing that user-centred design improves product usefulness and usability, pointed out a lack of effectiveness measures to evaluate the success of a user-centred design effort.

\subsection{Participatory design}
The emergence of participatory design, according to \citet{bannon2018introduction}, was motivated by the need to involve users in the process of design, and to help mitigate situations of divergent ideas through democratic decision-making processes. Participatory design allows the involvement of the users of the system in the design process and gives stakeholders the opportunity to learn more about other stakeholders' views \cite{unger2013designing}.
Often participatory design is referred to in discussions of user-centered design, so the difference between user-centered design and participatory design has become quite unclear \cite{bannon2018introduction}. Yet, participatory design is sometimes discussed alongside user-centred design, as a user-centred approach \cite{leng2018designing}.

\citet{dearnley1983favour}identified two main reasons why participation in design emerged: firstly, as a means of mitigating the problem of conventional approaches that do not work; and secondly, because of ethical issues, claiming that end-users should have some influence over the environment in which they work.

Arguing that the design paradigm of participatory design is constructivism,\citet{spinuzzi2005methodology} assumes that knowledge exists in a certain context, not just in a person’s mind. The author identified three stages of participatory design research: initial exploration of work, discovery processes and prototyping. 
\begin{itemize}
    \item \textit{Initial exploration of work}: this is the stage where users and designers meet to become familiar with how users work with each other. The used research methods are ethnographics and include observations, interviews and visits to the organisation.
    \item \textit{Discovery processes}: during this stage, the design stakeholders use different techniques to improve their understanding, prioritize the organization of work, and construct a future vision for the workplace. The used methods include organisational games, role-play games, organisational toolkits, future workshops, storyboarding, and workflow models and interpretation sessions.
    \item \textit{Prototyping}: Here, users and designers form technological artifacts in an iterative manner, to fit into the future vision created in the second stage. This is achieved using different methods including mock-ups, paper prototyping, and cooperative prototyping.
\end{itemize}


\subsection{Human-centered design approach in LAD}
Within the learning analytics community, designers and researchers are increasingly turning to human-centred design methods to adopt practical design techniques and knowledge to explore the design cycle of a learning analysis project. In the current literature on learning analysis, there are calls for theory-based and participatory approaches. \citet{alhadad2018visualizing}  noted that data visualization design should be grounded in theories of cognitive learning. For example, attention to issues of cognitive loading, attention, human perception and data literacy has direct design implications, such as avoiding visual clutter and breaking down data into segments that can be interpreted on the screen. This perspective is consistent with what data visualization researchers understand to be visual coding and interaction design decisions \citep{munzner2014visualization}, where choices regarding graphic types, interface components such as text and placement of visual markers, and interactive elements align with what is known about cognition and perception.
However, as noted by \citet{dollinger2018co}, learning analytics researchers often work in applied contexts with real educators and educational settings, where it quickly becomes apparent that interface-level design decisions alone are not sufficient to productively promote effective use of visual dashboards. 

\subsection{Participatory design}
\textit{Co-design} derived from user-centered design as a particular case of co-creation where designers who are trained in creativity work together with non-designers during the design process. \cite{Sanders2008} it as a process where designers and non-designers work together creatively ideally in all stages of design. A main claim of codesign is that it remedies design’s conventional exclusion of people from design outcomes that affect them \cite{taffe2015hybrid}.

\cite{siu2003users} describes the co-design process as a facilitator that directs the level of end-user awareness of design choices, providing professional advice and guidance on the consequences of alternative choices while end-users provide advice and contribute to the contextual experience.  Research suggests that including end-users in the design process ensures appropriate design responses that are aligned with end-user requirements \cite{taffe2015hybrid}. 


To glean as much information as possible about the context and needs of users, 
a common strategy is to use participatory design techniques where end-users are directly invited to participate in the design process. 
Participatory Design for Learning is a growing field with its own history, philosophies, and body of techniques and methods \cite{disalvo2017participatory}. In practical terms, the LA researchers used a subset of techniques such as pre-interviewing end-users about 
their needs to derive design ideas \cite{xhakaj2016teachers}, employing practitioner-partners as informant designers who give feedback on 
designs \cite{fiorini2018application}, and engaging teachers throughout the prototyping process itself to create classroom analytics \cite{holstein2018classroom}.


\subsection{CoDesign approaches of LADs}
In educational settings, designers and developers commonly rely on teams of visual designers, business specialists and educational theorists who produce educational innovations (Herold, 2015). I
\citet*{mccoll2012health} define cocreation, or codesign, as \textit{the benefits realized from the integration of resources through activities and interactions with collaborators}. historicall, it is thus originally a marketing and management strategy, incorporating stakeholder and user resources into the design, process and analysis of services, breaking down the traditional boundaries between producers and users. Co-creation can benefit the current design of learning
analytics in several key ways.


The benefits of co-designing a dashboard with the collaboration of multiple stakeholders with diverse perspectives are numerous. A dashboard design team should include education administrators, technology experts, data managers and, most importantly, learners. A co-design approach encourages alignment among multiple stakeholders with the purpose and main goal of the dashboard, and ensures consistent implementation by anticipating obstacles or potential difficulties in adopting the system, and increases innovation capacity and efficiency \citet{boscardin2018twelve}. A participatory approach also facilitates opportunities for consensus building and the creation of shared mental models for key learner outcomes, as well as the appropriate interpretation and use of evaluation data by multiple stakeholders. As pointed out by \citet{schuler1993participatory}, a co-design approach reflects a participatory design movement that emphasizes mutual learning, with users learning about the opportunities and constraints of the technology while designers learn about user practices and perspectives.

\subsection{Techniques for codesign in LA}


\section{Related Work}
Co-design techniques have started to attract the attention of some researchers and
practitioners within the learning analytics community, especially for learner
consultation (Holstein, McLaren,  Aleven, 2017; McPherson, Tong, Fatt,  Liu, 2016;
Roberts., Howell.,  Seaman., 2016), to understand privacy concerns (Slade 
Prinsloo, 2015), for tailoring support for learners (Madeline Huberth, Nicole Michelotti,
 McKay, 2013), for designing learning activities (Könings, Seidel,  van Merriënboer,
2014) and for designing dashboards (Corrin  Barba, 2015).
\subsection{LA-DECK: a card-based learning analytics co-design tool \cite{Alvarez2020}}
\subsection{\citet{prestigiacomo2020learning}}
\subsection{\citet{prestigiacomo2020data}}
\subsection{\citet{prestigiacomo2020learning}}
\subsection{\citet{prestigiacomo2020learning}}
\subsection{\citet{prestigiacomo2020learning}}
\subsection{\citet{prestigiacomo2020learning}}


\section{Paddle}
\subsection{Design Thinking}
Design Thinking is a way of embedding the notion of iterative design into the
design process. Coming up with ideas, building and testing in a short period of time is
what Design Thinking brings to the table in comparison to other iterative processes
(Ellingsen, 2016). In terms of innovation through iterations, Design Thinking specifies
three main stages (Koh, Chai, Wong,  Hong, 2015): understanding, creation and
delivery (see inner square in Figure 1). The iterative process starts with creating
empathy and defining goals/expectations from stakeholders. This first part of the
process can be exploratory and in some cases may set the tone of the project for the
next stages (IDEO, 2016). 

\subsection{}

\subsection{}

\subsection{}

\subsection{}
\section{Conclusion}


\bibliographystyle{plainnat}
\bibliography{references}
\end{document}
