\documentclass{article}
\usepackage[utf8]{inputenc}

\title{CoDesign of Learning Analytics Dashboards}
\author{...}
\date{January 2021}
 
\usepackage{natbib}
\usepackage{graphicx}

\begin{document}

\maketitle

\section{Introduction}
\dots
Learning analytics dashboards are designed to use learners' traces to present the computed indicators and other visual elements in a clear and intuitive way \citep{Brouns2015}.
They have emerged as applications for visualizing and interacting with data collected in a learning environment in various forms \citep{Ramos2015}. \citet{Steiner2014} referred to them as ``visualizations of learning traces''. For \citet{Yoo2015}, a learning dashboard is ``a display which visualizes the results of educational data mining in a useful way''. 
\citet{Schwendimann2017} identified a lack of an agreed and shared definition and thus proposed the following:
\begin{quotation}
	``\textit{%
		A learning dashboard is a single display that aggregates different indicators
		about learner(s), learning process(es) and/or learning context(s) into one or
		multiple visualizations.
	}'' \citep{Schwendimann2017}
\end{quotation}
Learning dashboards provide interactive, historical, customized and analytical displays that are based on the results of  analyzing learning data \citep{Park2015, Kim2015}. By implementing visual and interactive analytics, they amplify human natural abilities to detect patterns, establish connections and make inferences. The produced visual outputs can significantly highlight aspects of interest from the mined and discovered knowledge~\citep{Duval2011}. 

Learning dashboards are suitable for online,  face-to-face, and blended learning \citep{Verbert2013}. They can target different stakeholders: administrators, instructors, learners or all of them. 
Within a single display, indicators and visualizations about learners, learning processes and contexts are rendered using different shapes, from plain text to visual elements (e.g., tables, spreadsheet charts, scatterplot, 3D representations) to complex artifacts such as alerts and notifications that prompt interventions~\citep{Few2006, Podgorelec2011, Schwendimann2017}. 
Currently, they are increasingly deployed as a meaningful component in learning analysis systems. For instance, they are currently used in studying progression through courses \citep{Nicholson2012}, learners level of attainment~\citep{Gutierrez2012}, and learners' engagement from the cognitive and behavioral perspectives \citep{Carrillo2017}.
Despite being fairly recent educational tools, the research found many benefits of using learning dashboards to improve learning performance \citep{Arnold2012} and to increase learners' motivation    \citep{Verbert2013, Wise2016}.

\dots

\section{Leanring analytics Dashboards}
According to \citet{bennett2019four}, we can distinguish  three types of learner dashboards: predictive, modeler and descriptive. 
\begin{itemize}
    \item Predictive types use machine learning based on computer algorithms that rely on a series of trace data to predict the likelihood that the students outcomes. Many works in this category (i.e, \citep{arnold2014exercise}) use these algorithms to produce at risk rating for individual students.
    \item For example, aspects of a student's online behaviours such as communication, initiative, presence that have been derived from the number of messages, the number of comments in response to others' messages, time spent online, etc., are not included in this section. 
    \item The third type of dashboard is descriptive and displays past learning behaviours
\end{itemize}

\section{Designing leanring analytics dashboards}
There is a growing body of literature on learner dashboard design, and several literature reviews covering the field. A comprehensive literature review by \citet{bodily2017trends} involved identifying and analyzing 93 papers with respect to functionality, data sources, design analysis, student perceptions, and measured effects. They found that the clear majority of the papers focused on technical features such as data sources, functional aspects (e.g., visualization data, what the dashboard was intended to do) while only two papers reported on how students' behaviours were changed as a result of using the Learner Dashboards and only one-third (34 of 93) focused on students' perceptions of the design.



\section{Conclusion}


\bibliographystyle{plainnat}
\bibliography{references}
\end{document}
